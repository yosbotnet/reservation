\documentclass[12pt,a4paper]{report}

% Packages
\usepackage[utf8]{inputenc}
\usepackage[T1]{fontenc}
\usepackage{graphicx}
\usepackage{float}
\usepackage{booktabs}
\usepackage{amsmath}
\usepackage{hyperref}
\usepackage[italian]{babel}
\usepackage[left=2.5cm,right=2.5cm,top=2.5cm,bottom=2.5cm]{geometry}

% Rename table of contents
\renewcommand{\contentsname}{Indice}

\title{{\Huge Elaborato per il corso di Basi di Dati}\\[1cm]
       {\Large A.A 2024/2025}\\[1cm]
       {\Large Progetto di una base di dati per la gestione di una clinica chirurgica}}
       
\author{Yosberto Baro Carbonell\\[0.5cm]
        yosberto.barocarbonell@studio.unibo.it\\
        Matricola: 0001090988
        }
\date{}

\begin{document}

\maketitle

\tableofcontents

\chapter{Analisi dei requisiti}

\section{Intervista}
La seguente descrizione riporta in linguaggio naturale i requisiti per il nostro sistema informativo:

"Si vuole realizzare un sistema informatico di supporto alla gestione di una clinica chirurgica. Il sistema deve gestire tutte le attività relative alla programmazione degli interventi chirurgici, all'organizzazione delle operazioni e alla gestione delle cure post-operatorie, garantendo un'esperienza efficiente sia per i pazienti che per il personale medico.

Il sistema prevede che ogni paziente debba registrarsi fornendo i propri dati personali (nome, cognome, codice fiscale, data di nascita, email, telefono) e le informazioni sanitarie di base come gruppo sanguigno e allergie. Una volta registrato, il paziente può richiedere una visita con un determinato dottore, scegliendo tra le date in cui il dottore è disponibile.

La clinica impiega diversi dottori, ciascuno con le proprie specializzazioni. Ogni dottore ha un profilo (nome, cognome, numero di registrazione e aree di competenza) e gestisce il proprio calendario per indicare la disponibilità settimanale.

Gli interventi chirurgici sono categorizzati per tipologia e complessità. Ogni tipo di intervento richiede risorse specifiche, come una sala operatoria e attrezzature specializzate, e ha una durata standard stimata. Dopo l'operazione, il sistema tiene traccia delle cure post-operatorie, inclusi i farmaci somministrati e le informazioni sulle attività da evitare.

Il sistema offre funzionalità diverse a seconda del tipo di utente:

Area Pubblica:
- Registrazione nuovo account paziente
- Richiesta visita con dottore specifico
- Visualizzazione appuntamenti chirurgici
- Visualizzazione protocolli post-operatori assegnati

Area Dottori:
- Accettazione visite pazienti
- Creazione appuntamento chirurgico
- Gestione calendario disponibilità
- Visualizzazione interventi programmati
- Accesso informazioni pazienti assegnati

Area Amministrativa:
- Gestione anagrafica utenti
- Gestione sale operatorie e attrezzature
- Gestione catalogo interventi offerti
- Gestione inventario attrezzature

Il sistema deve inoltre fornire le seguenti informazioni:
- Tasso di occupazione delle sale operatorie negli ultimi 3 mesi
- Statistiche di confronto tra durate stimate ed effettive degli interventi
- Numero di operazioni eseguite per tipologia nell'ultimo anno
- Lista dei dottori con il maggior numero di interventi completati con successo"

\section{Estrazione dei concetti principali}
Dal testo dell'intervista possiamo estrarre i seguenti concetti principali:

\begin{table}[H]
\caption{Tabella dei concetti principali}
\begin{tabular}{p{3cm}p{12cm}}
\toprule
\textbf{Termine} & \textbf{Descrizione} \\
\midrule
Paziente & Utente che si registra per ricevere cure mediche, caratterizzato da dati personali e informazioni sanitarie \\
Dottore & Professionista medico con specializzazioni specifiche che esegue visite e interventi \\
Visita & Appuntamento tra paziente e dottore che precede un possibile intervento \\
Intervento & Operazione chirurgica programmata con specifiche risorse e tempistiche \\
Sala Operatoria & Ambiente dove si svolgono gli interventi, dotato di attrezzature specifiche \\
Attrezzatura & Strumenti e macchinari necessari per gli interventi \\
Cura Post-operatoria & Protocollo di cure da seguire dopo l'intervento \\
\bottomrule
\end{tabular}
\end{table}

% Qui andrebbe inserito lo schema scheletro iniziale
\begin{figure}[H]
\caption{Schema scheletro iniziale delle entità principali}
\centering
% [INSERIRE IMMAGINE schema-scheletro.pdf]
% Lo schema dovrebbe mostrare le entità principali (Paziente, Dottore, Intervento, Sala) 
% e le loro relazioni fondamentali
\end{figure}

Prima di procedere con la progettazione dettagliata, è necessario evidenziare alcune ambiguità e vincoli non espressi nel testo:

\begin{itemize}
\item Il sistema di prenotazione delle visite deve considerare sovrapposizioni?
\item Come gestire le priorità degli interventi?
\item Come gestire le sostituzioni dei dottori in caso di emergenza?
\item Quali sono i criteri per l'assegnazione delle sale operatorie?
\item Come gestire le modifiche agli appuntamenti già programmati?
\end{itemize}

Queste ambiguità verranno risolte durante la fase di progettazione, implementando opportuni vincoli nel database.

\chapter{Progettazione concettuale} 
\section{Schema scheletro}
\section{Raffinamenti proposti}
\section{Schema concettuale finale}

\chapter{Progettazione logica}
\section{Stima del volume dei dati}
\section{Descrizione delle operazioni principali e stima della loro frequenza}
\section{Schemi di navigazione e tabelle degli accessi}
\section{Analisi delle ridondanze}
\section{Raffinamento dello schema}
\section{Traduzione di entità e associazioni in relazioni}
\section{Schema relazionale finale}
\section{Traduzione delle operazioni in query SQL}

\chapter{Progettazione dell'applicazione}
\section{Descrizione dell'architettura dell'applicazione realizzata}
\section{Screenshot dell'interfaccia utente}

\end{document}