\documentclass[12pt,a4paper]{report}

% Packages
\usepackage[utf8]{inputenc}
\usepackage[T1]{fontenc}
\usepackage{graphicx}
\usepackage{float}
\usepackage{booktabs}
\usepackage{amsmath}
\usepackage{hyperref}
\usepackage[italian]{babel}
\usepackage[left=2.5cm,right=2.5cm,top=2.5cm,bottom=2.5cm]{geometry}

% Rename table of contents
\renewcommand{\contentsname}{Indice}

\title{{\Huge Elaborato per il corso di Basi di Dati}\\[1cm]
       {\Large A.A 2024/2025}\\[1cm]
       {\Large Progetto di una base di dati per la gestione di una clinica chirurgica}}
       
\author{Yosberto Baro Carbonell\\[0.5cm]
        yosberto.barocarbonell@studio.unibo.it\\
        Matricola: 0001090988
        }
\date{}

\begin{document}

\maketitle

\tableofcontents

\chapter{Analisi dei requisiti}

\section{Intervista}
La clinica chirurgica BaroMed© richiede un sistema informatico per la gestione delle attività relative alla programmazione e all'esecuzione degli interventi chirurgici. Il database di informazioni verrà acceduto sia dai pazienti, tramite un'area pubblica, sia dai dottori e dagli amministratori del sistema attraverso aree riservate. Il sistema deve poter memorizzare la lista degli utenti registrati, ciascuno dei quali avrà un proprio account associato a un profilo specifico.

Un paziente deve innanzitutto registrarsi fornendo i propri dati anagrafici (nome, cognome, codice fiscale, data di nascita) e i recapiti (email, telefono), oltre alle informazioni sanitarie di base come gruppo sanguigno e allergie. Una volta registrato, il paziente può prenotare una visita con un determinato dottore, scegliendo tra le date disponibili nel calendario del medico.

Ogni dottore della clinica ha un proprio profilo professionale che include numero di registrazione all'albo e specializzazioni. Il dottore gestisce il proprio calendario settimanale indicando i giorni e gli orari in cui è disponibile per visite ed interventi. Durante una visita, il dottore può programmare un intervento chirurgico se necessario.

Gli interventi sono classificati per tipologia e livello di complessità (bassa, media, alta). Ogni tipo di intervento ha requisiti specifici in termini di:
\begin{itemize}
\item Sala operatoria da utilizzare
\item Attrezzature specialistiche necessarie
\item Durata standard stimata
\item Costo della procedura
\end{itemize}

Al termine di ogni intervento viene registrato l'esito e viene definito un protocollo post-operatorio che include:
\begin{itemize}
\item Farmaci da assumere con relative posologie
\item Attività fisiche da evitare
\item Controlli di follow-up da effettuare
\end{itemize}

\begin{table}[H]
\caption{Rilevamento delle ambiguità e correzioni proposte}
\begin{tabular}{p{3cm}p{6cm}p{6cm}}
\toprule
\textbf{Riferimento} & \textbf{Ambiguità} & \textbf{Correzione} \\
\midrule
Disponibilità dottori & Non è specificato come gestire le sovrapposizioni di orari & Implementare un sistema di controllo che impedisca la sovrapposizione di appuntamenti per lo stesso dottore \\
\midrule
Prenotazione sale & Non è chiaro il criterio di assegnazione delle sale operatorie & Le sale vengono assegnate in base alla tipologia di intervento e alla disponibilità delle attrezzature necessarie \\
\midrule
Modifiche appuntamenti & Non è specificato come gestire cambi di data/ora & Permettere modifiche solo se non creano conflitti con altri appuntamenti già programmati \\
\midrule
Sostituzioni & Non è definita la gestione delle emergenze/sostituzioni & Implementare un sistema di reperibilità dei dottori per specializzazione \\
\midrule
Durata interventi & Non è specificato come gestire gli sforamenti temporali & Prevedere dei margini di sicurezza tra interventi consecutivi \\
\bottomrule
\end{tabular}
\end{table}

\section{Estrazione dei concetti principali}
A seguito della lettura e comprensione dei requisiti, si procede con l'estrazione dei concetti fondamentali e delle loro relazioni:

\begin{table}[H]
\caption{Glossario dei termini}
\begin{tabular}{p{3cm}p{12cm}}
\toprule
\textbf{Termine} & \textbf{Descrizione} \\
\midrule
Utente & Entità base che rappresenta qualsiasi utilizzatore del sistema (paziente, dottore, amministratore) \\
Paziente & Estensione di Utente che include informazioni sanitarie specifiche \\
Dottore & Estensione di Utente con dati professionali e specializzazioni \\
Visita & Consultazione medica preliminare all'eventuale intervento \\
Intervento & Procedura chirurgica programmata con specifiche risorse e tempistiche \\
Sala Operatoria & Ambiente attrezzato dove si svolgono gli interventi \\
Attrezzatura & Strumentazione specialistica necessaria per gli interventi \\
Farmaco & Medicinale prescritto nel protocollo post-operatorio \\
Attività & Azione fisica da evitare durante la convalescenza \\
\bottomrule
\end{tabular}
\end{table}

\begin{figure}[H]
\caption{Schema scheletro iniziale}
\centering
% [INSERIRE IMMAGINE schema-scheletro.png]
% Diagramma E/R che mostra le entità principali e le loro relazioni:
% - Utente (specializzato in Paziente, Dottore, Admin)
% - Visita (collega Paziente e Dottore)
% - Intervento (collega Paziente, Dottore e Sala)
% - Protocollo post-operatorio
\end{figure}

Questo schema verrà raffinato nel capitolo successivo, aggiungendo attributi e vincoli specifici per ogni entità e relazione.

\chapter{Progettazione concettuale}

\section{Schema scheletro}
Lo sviluppo dello schema Entity-Relationship procederà per fasi successive, partendo da uno schema base che verrà progressivamente raffinato. La complessità del sistema richiede particolare attenzione nella modellazione di:

\begin{itemize}
\item La struttura gerarchica degli utenti del sistema
\item Le relazioni temporali tra visite e interventi
\item La gestione delle risorse (sale operatorie e attrezzature)
\item Il tracciamento delle cure post-operatorie
\end{itemize}

Nel primo schema scheletro si evidenziano le entità fondamentali e le loro relazioni principali, rimandando a successive fasi di raffinamento l'aggiunta di dettagli specifici e vincoli.

\begin{figure}[H]
\caption{Schema scheletro iniziale}
\centering
% TODO: Inserire schema scheletro base
\end{figure}

\section{Raffinamenti proposti}
Analizziamo ora le principali questioni progettuali da affrontare:

\subsection{Gestione degli utenti}
La prima questione riguarda la modellazione degli utenti del sistema. È necessaria una gerarchia che distingua tra pazienti e dottori, mantenendo però alcuni dati in comune (credenziali di accesso, dati anagrafici). Si propone quindi una generalizzazione dell'entità UTENTE.

\textbf{Problema}: Come gestire i diversi tipi di utenti mantenendo i dati comuni?

\textbf{Soluzione}: Si adotta una gerarchia di generalizzazione totale ed esclusiva (t,e) con l'entità UTENTE come genitore e le entità PAZIENTE e DOTTORE come figlie. Questo permette di:
\begin{itemize}
\item Centralizzare la gestione delle credenziali e dei dati anagrafici
\item Differenziare gli attributi specifici per tipo di utente
\item Garantire che ogni utente sia esattamente di un tipo
\end{itemize}

\textbf{Problema}: Come modellare le competenze dei dottori?

\textbf{Soluzione}: Si introduce l'entità SPECIALIZZAZIONE collegata a DOTTORE attraverso una relazione molti-a-molti, permettendo ad ogni dottore di avere più specializzazioni e ad ogni specializzazione di essere posseduta da più dottori.

\subsection{Gestione delle visite}
Il sistema deve tracciare le visite mediche, che costituiscono il punto di partenza per eventuali interventi.

\textbf{Problema}: Come collegare visite e interventi mantenendo la storia?

\textbf{Soluzione}: L'entità VISITA viene modellata con:
\begin{itemize}
\item Collegamenti diretti a paziente e dottore
\item Data e ora della visita
\item Possibilità di programmare un intervento
\end{itemize}

[Continua con ulteriori raffinamenti...]
\section{Schema concettuale finale}

\chapter{Progettazione logica}
\section{Stima del volume dei dati}
\section{Descrizione delle operazioni principali e stima della loro frequenza}
\section{Schemi di navigazione e tabelle degli accessi}
\section{Analisi delle ridondanze}
\section{Raffinamento dello schema}
\section{Traduzione di entità e associazioni in relazioni}
\section{Schema relazionale finale}
\section{Traduzione delle operazioni in query SQL}

\chapter{Progettazione dell'applicazione}
\section{Descrizione dell'architettura dell'applicazione realizzata}
\section{Screenshot dell'interfaccia utente}

\end{document}